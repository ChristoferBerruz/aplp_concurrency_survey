\section{Shared-Memory Model\label{sec:shared_memory}}
A shared-memory concurrency model is any model in which units of computation,
processes or threads, communicate with each other using shared memory.
Communication is done by reading and writing to a shared object. Refer to Figure
\ref{fig:shared_memory} for a visual representation of the shared-memory model.

Note that the definition \textit{memory} can be loosened depending on the application.
For example, \cite{mitConcurrency} illustrates the following scenarios
where shared-memory concurrency models are applicable:

\begin{itemize}
    \item A, B can be two processes communicating using the same process memory.
    \item A, B can be two programs running on the same hardware, but using the same filesystem.
    \item A, B can be two running threads of some computer program sharing the same object.
\end{itemize}

\begin{figure}
    \centering
    \begin{tikzpicture}[node distance=1cm and 1.5cm, >=Stealth]

        % Shared memory bar
        \draw[thick] (0,0) rectangle (11,1);
        \node at (9,0.5) {Shared memory};
      
        % Memory blocks
        \fill[blue!50] (0.3,0.2) rectangle (1.3,0.8);
        \fill[red!70] (2.9,0.2) rectangle (3.9,0.8);
        \fill[blue!50] (5.5,0.2) rectangle (6.5,0.8);
      
        % Processes A and B
        \node[draw, minimum width=1cm, minimum height=1cm] (A) at (1.3,3) {A};
        \node[draw, minimum width=1cm, minimum height=1cm] (B) at (5.7,3) {B};
      
        % Arrows from processes to memory blocks
        \draw[->] (A) -- (0.8,1);
        \draw[->] (A) -- (3.4,1);
        \draw[->] (B) -- (3.4,1);
        \draw[->] (B) -- (6.0,1);
      
      \end{tikzpicture}
      \caption{
        Shared-memory concurrency model. 
        Processes A and B communicate by reading and writing to shared memory blocks.
      }
      \label{fig:shared_memory}
\end{figure}