\section{Conclusion\label{sec:conclusion}}
In this survey we described the Shared-Memory (SM),
Actor-Model (AM), and Software Transactional Memory (STM)
concurrency models. We focused on
the motivation, design, implementation, challenges,
and advantages of each model. The goal of this survey
was to describe these models to help readers
understand the complexities, pitfalls, and options
available when writing concurrent programs.

The Shared-Memory (SM) concurrency model
is simple to implement but prone to challenges like deadlocks,
nondeterminism, and debugging complexity.
While synchronization primitives and design patterns
such as mutexes and turnstiles help manage critical sections,
they add complexity and rely heavily on programmers' correct usage.
Bugs in SM systems are often difficult to reproduce due to
interleaving issues and varying memory consistency models
depending on hardware, compiler, or runtime.
Efforts to improve determinism, such as replay systems and
deterministic frameworks, aim to address these challenges,
making SM concurrency more predictable and manageable.

Although the actor model provides a clear conceptual framework encapsulating state 
within independent actors and communicating solely via asynchronous messages, 
it is not without its challenges. Message ordering is inherently 
non-deterministic, making subtle bugs difficult to reproduce. 
Developers must experiment with supervision hierarchies, 
mailbox sizing, back-pressure strategies, and failure propagation 
patterns to ensure correct, performant behavior. Each of these 
constructs, while powerful, adds cognitive overhead, and integrating 
actor-based designs with legacy synchronous I/O or shared-memory libraries 
can introduce additional friction. Under heavy load, the serialization and 
queueing of messages may also incur performance overhead that requires careful 
tuning of dispatchers and routing strategies. In production, mature frameworks 
like Akka on the JVM, show how the actor model can yield highly available, 
fault-tolerant systems with dynamic scaling and location transparency. Their 
ecosystems offer robust tooling for monitoring, clustering, and supervised 
restarts, and their communities value the model’s composability and resilience.

STM remains an elegant concurrency model and continues to be supported (implemented), notably by Haskell and Clojure.
In these environments, immutability and strong typing compliment STM’s complexity,
and the smaller and academically minded community is comfortable with its cost
in exchange for composability and safety. In more conventional and corporate
backed ecosystems like .NET, Python or JVM, STM has largely been sidelined
in favor of established models like that of task-actor-message models;
ones that align with the easier idioms and tooling.
Performance considerations, difficulty with I/O and legacy code,
and the lack of a “killer app” cemented STM’s decline.
Each community made reasonable choice: either optimize
around STM or pivot to alternatives better suited.
It is not a Swiss Army knife, but a scalpel for specific use cases.